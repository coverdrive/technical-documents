%%%%%%%%%%%%%%%%%%%%%%%%%%%%%%%%%%%%%%%%%
% Beamer Presentation
% LaTeX Template
% Version 1.0 (10/11/12)
%
% This template has been downloaded from:
% http://www.LaTeXTemplates.com
%
% License:
% CC BY-NC-SA 3.0 (http://creativecommons.org/licenses/by-nc-sa/3.0/)
%
%%%%%%%%%%%%%%%%%%%%%%%%%%%%%%%%%%%%%%%%%

%----------------------------------------------------------------------------------------
%	PACKAGES AND THEMES
%----------------------------------------------------------------------------------------

\documentclass[handout]{beamer}

\mode<presentation> {

% The Beamer class comes with a number of default slide themes
% which change the colors and layouts of slides. Below this is a list
% of all the themes, uncomment each in turn to see what they look like.

%\usetheme{default}
%\usetheme{AnnArbor}
%\usetheme{Antibes}
%\usetheme{Bergen}
%\usetheme{Berkeley}
%\usetheme{Berlin}
%\usetheme{Boadilla}
%\usetheme{CambridgeUS}
%\usetheme{Copenhagen}
%\usetheme{Darmstadt}
%\usetheme{Dresden}
%\usetheme{Frankfurt}
%\usetheme{Goettingen}
%\usetheme{Hannover}
%\usetheme{Ilmenau}
%\usetheme{JuanLesPins}
%\usetheme{Luebeck}
\usetheme{Madrid}
%\usetheme{Malmoe}
%\usetheme{Marburg}
%\usetheme{Montpellier}
%\usetheme{PaloAlto}
%\usetheme{Pittsburgh}
%\usetheme{Rochester}
%\usetheme{Singapore}
%\usetheme{Szeged}
%\usetheme{Warsaw}

% As well as themes, the Beamer class has a number of color themes
% for any slide theme. Uncomment each of these in turn to see how it
% changes the colors of your current slide theme.

%\usecolortheme{albatross}
%\usecolortheme{beaver}
%\usecolortheme{beetle}
%\usecolortheme{crane}
%\usecolortheme{dolphin}
%\usecolortheme{dove}
%\usecolortheme{fly}
%\usecolortheme{lily}
%\usecolortheme{orchid}
%\usecolortheme{rose}
%\usecolortheme{seagull}
%\usecolortheme{seahorse}
%\usecolortheme{whale}
%\usecolortheme{wolverine}

%\setbeamertemplate{footline} % To remove the footer line in all slides uncomment this line
%\setbeamertemplate{footline}[page number] % To replace the footer line in all slides with a simple slide count uncomment this line

%\setbeamertemplate{navigation symbols}{} % To remove the navigation symbols from the bottom of all slides uncomment this line
}

\usepackage{graphicx} % Allows including images
\usepackage{booktabs} % Allows the use of \toprule, \midrule and \bottomrule in tables
\usepackage{cool}
\usepackage{tikz}
\usepackage{amsmath}
\usepackage{xcolor}
\usepackage{hyperref}
\usepackage{bm}

\DeclareMathOperator*{\argmax}{argmax}
\DeclareMathOperator*{\argmin}{argmin}
\usetikzlibrary{positioning}

%----------------------------------------------------------------------------------------
%	TITLE PAGE
%----------------------------------------------------------------------------------------

\title[AI in Retail]{AI Opportunities in Retail} % The short title appears at the bottom of every slide, the full title is only on the title page

\author{Ashwin Rao} % Your name
\institute[Stanford] % Your institution as it will appear on the bottom of every slide, may be shorthand to save space
{Stanford University
 % Your institution for the title page
}

\date{} % Date, can be changed to a custom date

\begin{document}
\begin{frame}
\titlepage % Print the title page as the first slide
\end{frame}

% \begin{frame}
% \frametitle{Overview} % Table of contents slide, comment this block out to remove it
% \tableofcontents % Throughout your presentation, if you choose to use \section{} and \subsection{} commands, these will automatically be printed on this slide as an overview of your presentation
% \end{frame}


\begin{frame}
\frametitle{A bit about me}
\pause
\begin{itemize}[<+->]
\item Chief AI Officer at QXO (Distribution of Building Materials)
\item Adjunct Professor, \href{https://icme.stanford.edu/}{\underline{\textcolor{blue}{Applied Mathematics (ICME)}}}, Stanford University
\item Past: VP of AI at Target Corporation
\item Past: MD at Morgan Stanley, Trading Strategist at Goldman Sachs
\item Leading Stanford's \href{https://mcf.stanford.edu/}{\underline{\textcolor{blue}{Mathematical \& Computational Finance program}}}
\item Research \& Teaching in: {\em RL and it's applications in Finance \& Retail}
\item Book:  \href{https://www.amazon.com/Foundations-Reinforcement-Learning-Applications-Finance/dp/1032124121}{\underline{\textcolor{blue}{Foundations of RL with Applications in Finance}}}
\item Teaching faculty in HAI for retail companies
\item Today I will talk about AI opportunities in Retail
\end{itemize}
\end{frame}

\begin{frame}
\frametitle{Overview of AI Opportunities}
\pause
\begin{itemize}[<+->]
\item Deep Learning-based Demand Forecasting
\item Operations: Inventory, Network, S\&OP, Transportation, Logistics
\item Commercial: Assortment, Presentations, Pricing
\item Shopping Agent: Agentic AI, Personalization
\item Employee Productivity: AI Agents serving as co-pilots/assistants
\end{itemize}
\end{frame}

\begin{frame}
\frametitle{Demand Forecasting Requirements}
\pause
\begin{itemize}[<+->]
\item Forecasting at various granularities of <SKU, Location, Time>
\item Combine “predictive” versus “reactive” capabilities
\item Understand SKU substitutability and complementarity
\item Incorporating ``tribal'' knowledge into forecasting

\end{itemize}
\end{frame}

\begin{frame}
\frametitle{Deep Learning can deliver on these requirements}
\pause
\begin{itemize}[<+->]
\item Investment in Data Engineering
\item Internal+External Data, Macro+Local Data, Real-time streaming
\item Granular Forecasts targeted towards Operational Control
\item Granular: Little data, highly uncertain forecasts, local factors
\item Coarse Forecasts targeted towards Planning and Design
\item Coarse:  Depends on many external factors/macro factors
\item Hierarchical forecasting blends macro effects with local effects
\item “predictive” is core, “reactive” is an update upon event reception 
\item {\em Transformers} blend structural patterns with regime shifts
\item {\em Embeddings} capture similarities and complementarities
\item System for capturing ``tribal'' knowledge in data systems
\end{itemize}
\end{frame}


\begin{frame}
\frametitle{Similar/Substitutable Products identified with {\em Embeddings}}
\pause
\begin{itemize}[<+->]
\item Can we automate identification of similarity/substitutability?
\item Similarity by Visuals or by Descriptions or by Customer Interest
\item Throw this heterogeneous data into deep neural network learning
\item Deep inside the neural network, we find encodings of these products
\item Similar/Substitutable products have similar encodings
\item The technical term for these encodings is \href{https://developers.google.com/machine-learning/crash-course/embeddings/video-lecture}{\underline{\textcolor{blue}{Embeddings}}}
\item Embeddings are low-dimensional numerical representations ({\em Vectors})
\item Capturing the most important features of products
\item Captures various relationships between products, eg: complementarity
\item Based on product images, descriptions, customer interest
\item Embeddings can be used as ML features for transfer learning
\item Embedding vectors have powerful algebraic/geometric properties
\end{itemize}
\end{frame}

\begin{frame}
\frametitle{Book Embeddings flattened to 2-D Vectors}
\includegraphics[width=12cm, height=8cm]{../supply_chain/BookEmbeddings.png}
\end{frame}

\begin{frame}
\frametitle{Transformers for learning the old and the new}
\pause
\begin{itemize}[<+->]
\item {\em Predictive:} Capture temporal patterns from long history
\item {\em Reactive:} Pay attention to recent shifts and trends
\item \href{https://en.wikipedia.org/wiki/Transformer_(machine_learning_model)}{\underline{\textcolor{blue}{Transformer Networks}}} eating the lunch of good old time-series stats
\item The core methodology is a technique called {\em Self-Attention}
\item It automatically identifies the relative importance of old versus new
\item Transformers are accurate and fast (highly parallelizable)
\item Core Tech in LLMs
\item Now being ported to numerical time-series applications
\item Embeddings blended with Transformers $\Rightarrow$ Practically Potent Combo
\item Not easy to pull off, requires considerable experimentation/tuning
\end{itemize}
\end{frame}

\begin{frame}
\frametitle{Capturing Stock Price Cycles/Trends with Transformers}
\includegraphics[width=11cm, height=7cm]{../supply_chain/StockPriceTransformer.jpg}
\end{frame}


\begin{frame}
\frametitle{Shopping Agent}
\pause
\begin{itemize}[<+->]
\item Current practice in e-commerce: Search \& Recommendations
\item Sometimes based on Product \& Customer Embeddings
\item e-commerce will soon transform with Agentic AI
\item Agents with deep knowledge of customer needs and preferences
\item Frequency, Basket-Size, Price, Health, Taste, Exploratory, Loyalty
\item Agent understands assortment, real-time inventory and prices
\item Agent converses and guides, counsels, optimizes, executes
\item Internal data curation to customize LLMs
\item Various real-time calls to smaller query/task agents
\item What is the objective to optimize? Multi-objectives? How exactly?
\item Dynamic objective construction
\item Plenty of testing needed to control hallucinations
\item Will take a few years to commercialize successfully
\end{itemize}
\end{frame}

\begin{frame}
\frametitle{Inventory Replenishment}
\pause
\begin{itemize}[<+->]
\item Let's consider a simple example of a single SKU in a store
\item The store experiences uncertain daily customer demand
\item The store can order daily from a supplier carrying infinite inventory
\item Cost associated with ordering, and order arrives in a few days
\item Inventory in the store incurs a daily {\em Holding Cost} (per unit)
\item Out-of-Stock incurs a {\em Stockout Cost} (per unit)
\item There's a notion of {\em Ordering Cost} (Labor/Transport)
\item When should one order? And in what quantity?
\item Traditional OR literature focused on closed-form ``solutions''
\item Approx closed-form solutions are often used in real businesses
\item In spite of the limitations of missing out on key real-world features
\end{itemize}
\end{frame}


\begin{frame}
\frametitle{Welcome to the Real-World}
\pause
\begin{itemize}[<+->]
\item Real-world not so simple - frictions, constraints, uncertainties
\item Supply can be constrained/uncertain
\item Space and Throughput constraints
\item Pricing/Marketing has a big influence on demand
\item New products, substitutable products complicate matters
\item In physical stores, there are {\em Presentation-Minimum} requirements
\item Often, products are shipped in casepacks
\item Irregular/Uncertain order frequency and lead times
\item Supply-Chain network might be {\em Multi-Echelon}
\item Inventory Count/Location is often uncertain
\item Cost of Damage, Theft, Misplacement, Spoilage
\item End-of-Season/Obsolescence/Spoilage requires Clearance/Salvage
\item All of this calls for a formal Decision Control framework
\end{itemize}
\end{frame}



\begin{frame}
\frametitle{The \href{https://en.wikipedia.org/wiki/Markov_decision_process}{\underline{\textcolor{yellow}{Markov Decision Process}}} Framework}
\includegraphics[width=12cm, height=7cm]{../finance/cme241/MDP.png}
\end{frame}

\begin{frame}
\frametitle{Components of the MDP Framework}
\pause
\begin{itemize}[<+->]
\item The {\em Agent} and the {\em Environment} interact in a time-sequenced loop
\item {\em Agent} responds to [{\em State}, {\em Reward}] by taking an {\em Action}
\item {\em Environment} responds by producing next step's (random) {\em State}
\item {\em Environment} also produces a (random) number we call {\em Reward}
\item Goal of {\em Agent} is to maximize {\em Expected Sum} of all future {\em Reward}s
\item By controlling the ({\em Policy} : {\em State} $\rightarrow$ {\em Action}) function
\item This is a dynamic (time-sequenced control) system under uncertainty
\item MDP framework enables modeling real-world Replenishment problems
\end{itemize}
\end{frame}

\begin{frame}
\frametitle{How a baby learns to walk}
\includegraphics[width=13cm, height=8cm]{../finance/cme241/BabyMDP.jpg}
\end{frame}

\begin{frame}
\frametitle{Many real-world problems fit this MDP framework}
\pause
\begin{itemize}[<+->]
\item Self-driving vehicle (speed/steering to optimize safety/time)
\item Game of Chess (Boolean {\em Reward} at end of game)
\item Inventory Replenishment to ensure high availability at low cost
\item Make a humanoid robot walk/run on difficult terrains
\item Manage an investment portfolio (covered in depth in my book)
\item Control a power station
\item Optimal decisions during a football game
\item Strategy to win an election (high-complexity MDP)
\end{itemize}
\end{frame}

\begin{frame}
\frametitle{Self-Driving Vehicle}
\includegraphics[width=13cm, height=8cm]{../finance/cme241/CarMDP.jpg}
\end{frame}

\begin{frame}
\frametitle{Real-World Replenishment as a Markov Decision Process}
\pause
\begin{itemize}[<+->]
\item MDP {\em State} is current Inventory Level at the store/warehouse
\item {\em State} also includes current in-transit inventory
\item {\em Action} is the multiple of casepack to order (or not order)
\item {\em Reward} function involves all of the costs we went over
\item State transitions governed by all of the uncertainties we went over
\item Solve: \href{https://en.wikipedia.org/wiki/Dynamic_programming}{\underline{\textcolor{blue}{Dynamic Programming}}} or
 \href{https://en.wikipedia.org/wiki/Reinforcement_learning}{\underline{\textcolor{blue}{Reinforcement Learning}}}
\item Curse of Dimensionality and Curse of Modeling $\Rightarrow$ RL
\end{itemize}
\end{frame}


\begin{frame}
\frametitle{How RL Works: Learning from Samples of Data}
\pause
\begin{itemize}[<+->]
\item RL incrementally learns from state/reward transitions data
\item Typically served by a simulator acting as a {\em Simulated Environment}
\item RL is a ``trial-and-error'' approach linking {\em Actions} to {\em Rewards}
\item Try different actions \& learn what works, what doesn't
\item Deep Neural Networks are typically used for function approximation
\item Big Picture: Sampling and Function Approximation come together
\item RL algorithms are clever about balancing  ``explore'' versus ``exploit''
\item Promise of modern A.I. is based on success of RL algorithms
\item Potential for automated decision-making in many industries
\item RL has many applications in Suppy-Chain, more broadly in Operations
\item RL covered in great detail in my book (theory, code and applications)
\end{itemize}
\end{frame}

\end{document}