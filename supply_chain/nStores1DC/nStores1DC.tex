
\documentclass[12pt]{amsart}
\usepackage{geometry} % see geometry.pdf on how to lay out the page. There's lots.
\geometry{a4paper} % or letter or a5paper or ... etc
% \geometry{landscape} % rotated page geometry

% See the ``Article customise'' template for come common customisations

\title{Joint Policy Optimization for $n$ Stores and 1 DC}
\author{}
\date{} % delete this line to display the current date

%%% BEGIN DOCUMENT
\begin{document}

\maketitle

\section{Network}

We consider $n$ stores indexed by $i = 1, 2, \ldots, n$. The $n$ stores are served inventory from a common DC. In general, we will index a DC variable with 0 when refering to the same variable that we might use for stores. The DC is served inventory from a vendor with infinite inventory. The lead time for inventory to move from DC to Store $i$ is $L_i$ ($1 \leq i \leq n$) and the lead time for inventory to move from Vendor to DC is $L_0$.

\section{Costs}
\begin{itemize}
\item The holding costs at the $n+1$ nodes ($n$ stores and 1 DC) are $h_i > 0$. $h_i$ is cost per unit of on-hand inventory before receiving inventory in an epoch.
\item The stockout cost (cost of missed sales) at store $i$ ($1 \leq i \leq n$) is $p_i > h_i$. $p_i$ is cost per unit of missed sales in an epoch.
\item The presentation-minimum at store $i$ ($1 \leq i \leq n$) is $PM_i \geq 0$. The cost of violation of presentation-mimimum at store $i$ ($1 \leq i \leq n$) is $v_i > h_i$ (also, $v_i < p_i$). $v_i$ is cost per unit of inventory below $PM_i$ before receiving inventory in an epoch.
\item The maximum number of units that node $i$ can carry is $M_i$, for all $0 \leq i \leq n$. The throwout cost at node $i$ is $c_i$ for all $0 \leq i \leq n$. $c_i$ is cost per unit of inventory above $M_i$ after receiving inventory in an epoch.
\item The cost of initiating $x$ units of inventory from parent of node $i$ to node $i$ ($0 \leq i \leq n$) is $\mathbb{I}_{x > 0} \cdot (K_i + J_i \cdot x)$
\end{itemize}

\section{Inventory}
\begin{itemize}
\item Denote on-hand inventory (a.k.a. Inventory Level) at node $i$ at the start of epoch $t$ as: $IL_{t,i} \geq 0$
\item Denote on-order inventory arriving at node $i$ in $k$ epochs ($1 \leq k \leq L_i$) at the start of epoch $t$ as $OO_{t,i,k} \geq 0$
\end{itemize}

\section{Inventory Movements}
Denote quantity of inventory movement initiated in epoch $t$ from the parent of node $i$ to node $i$ ($0 \leq i \leq n$) as $q_{t,i} \geq 0$. Node $i$ will receive that inventory of $q_{t,i}$ in epoch $t + L_i$. Denote $R_{t,i}$ as the inventory received in epoch $t$ at node $i$. Following the epoch of initiation of inventory movement $t$ and until the epoch of inventory receipt $t + L_i$, this quantity $q_i$ will appear as on-order $OO_{t+j,i,L_i-j+1}, 1 \leq j \leq L_i$. For the special case where $L_i = 0$, $R_{t,i} = q_{t,i}$ (Seqence of Events below illustrates that within an epoch, receipt of inventory happens after initiation of movement).

Demand at store $i$ ($1 \leq i \leq n$) in epoch $t$ is denoted by random variable $D_{t,i}$.

\section{States and Actions}
{\em State} $S_t$ in epoch $t$ is defined by the vector:
 $$[IL_{t,0}, OO_{t,0,1}, \ldots OO_{t,0, L_0}, IL_{t,1}, OO_{t,1,1}, \ldots, OO_{t,1,L_1}, \ldots, IL_{t,n}, OO_{t,n,1}, \dots, OO_{t,n, L_n}]$$
 {\em Action} $A_t$ in epoch $t$ is defined by the vector:
 $$[q_{t,0}, q_{t,1}, \ldots, q_{t,n}]$$


\section{Sequence of events in an epoch}

\begin{enumerate}
\item Observe {\em State} (simultaneous observation at all $n+1$ nodes of the inventory levels and at all $n+1$ arcs of the on-orders).
\item Perform {\em Action} (simultaneous initiation of movements of inventory at all $n+1$ arcs).
\item Calculate movement costs and holding costs for all nodes based on inventory levels at this point of the epoch
\item Receipt of inventory (simultaneous receipt of inventory at all $n+1$ nodes).
\item Calculate throwout cost at all nodes based on inventory levels at this point of the epoch. 
\item Occurrence of demand at stores (including missed sales, i.e., stockouts at the stores).
\item Calculate stockout costs and presentation violation costs for all stores based on inventory levels at this point of the epoch.
\end{enumerate}

\section{Equations defining Inventory Flow}
The following equations define the inventory flow in any epoch $t$:
$$R_{t,i} =
\begin{cases}
OO_{t,i,1} & \text{ if } L_i > 0\\
q_{t,i} & \text{ if } L_i = 0\\
\end{cases}
\mbox{ for all } t, \mbox{ for all } 0 \leq i \leq n$$
$$IL_{t+1,0} = \min(M_0, IL_{t,0} - \sum_{i=1}^n q_{t,i} + R_{t,0})$$
$$IL_{t+1,i} = \max(0, \min(M_i, IL_{t,i} + R_{t,i}) - D_{t,i}) \mbox{ for all } 1 \leq i \leq n$$
$$OO_{t+1,i,k} = OO_{t,i,k+1} \mbox{ for all } 0 \leq i \leq n, \mbox{ for all } 1 \leq k < L_i$$
$$OO_{t+1,i,L_i} = q_{t,i}  \mbox{ for all } 0 \leq i \leq n$$
$$\sum_{i=1}^n q_{t,i} \leq IL_{t,0} $$

\section{Cost Equations}
The {\em Reward} is defined by the following costs incurred in any epoch $t$:
\begin{itemize}
\item Movement Cost:
$$\sum_{i=0}^n \mathbb{I}_{q_{t,i} > 0} \cdot (K_i + J_i \cdot q_{t,i})$$
\item Holding Cost:
$$h_0 \cdot (IL_{t,0} - \sum_{i=1}^n q_{t,i}) + \sum_{i=1}^n h_i \cdot IL_{t,i}$$
\item Throwout Cost:
$$c_0 \cdot \max(0, IL_{t,0} - \sum_{i=1}^n q_{t,i} + R_{t,0} - M_0) + \sum_{i=1}^n c_i \cdot \max(0, IL_{t,i} + R_{t,i} - M_i)$$ 
\item Stockout Cost:
$$ \sum_{i=1}^n p_i \cdot \max(0, D_{t,i} - \min(M_i, IL_{t,i} + R_{t,i}))$$
\item Presentation-Violation Cost:
$$\sum_{i=1}^n v_i \cdot \max(0, PM_i - IL_{t+1,i})$$
\end{itemize}


\end{document}