
\documentclass[12pt]{amsart}
\usepackage{geometry} % see geometry.pdf on how to lay out the page. There's lots.
\geometry{a4paper} % or letter or a5paper or ... etc
% \geometry{landscape} % rotated page geometry

% See the ``Article customise'' template for come common customisations

\title{Constrained Dynamic Program for Backroom Minimization ensuring adequate Shelf Inventory}
\author{}
\date{} % delete this line to display the current date

%%% BEGIN DOCUMENT
\begin{document}

\maketitle

\section{Introduction}

$\mathbb{Z}$ refers to the set of integers, $\mathbb{R}$ refers to the set of real numbers. We will subscript $\mathbb{Z}$ and $\mathbb{R}$ to denote appropriate subsets of $\mathbb{Z}$ and $\mathbb{R}$. 
 
We consider a single store and single item served inventory from a supplier with infinite inventory and lead time of $L \in \mathbb{Z}_{\geq 0}$ epochs. Review period is assumed to be 1 epoch. There is a fixed capacity of $P \in \mathbb{Z}_{> 0}$ units for the item on the shelf (planogram) at the store. The item can only be replenished in multiples of $C \in \mathbb{Z}_{> 0}$ units ($C$ refers to the casepack size). Our goal is to identify a replenishment policy that minimizes the ``average backroom inventory'' (backroom inventory refers to the store inventory that is in excess of $P$) while ensuring that the shelf inventory in every epoch is at least a specified fraction $\alpha \in [0,1]$ of $P$ with probability at least $\beta \in [0,1]$.

\section{Inventory}
\begin{itemize}
\item Denote on-hand inventory (a.k.a. Inventory Level) at the store at the start of epoch $t$ as: $IL_t \in \mathbb{Z}$ (note: $IL_t$ is allowed to go negative if demand is unmet at the store, leading to back-ordering).
\item Denote on-order inventory arriving in $k$ epochs ($1 \leq k \leq L$) at the start of epoch $t$ as $OO_{t,k} \in \mathbb{Z}_{\geq 0}$
\end{itemize}

\section{Inventory Movements}
Denote number of casepacks of inventory ordered in epoch $t$ as $q_t \in \mathbb{Z}_{\geq 0}$. The store will receive that inventory of $q_t  C$ in epoch $t + L$. Denote $R_t \in \mathbb{Z}_{\geq 0}$ as the inventory received in epoch $t$. Following the epoch $t$ of inventory ordering and until the epoch $t+L$ of inventory receipt, this quantity $q_t C$ will appear in the flow equations (see below) as on-order $OO_{t+j,L-j+1}, 1 \leq j \leq L$. For the special case where $L = 0$, $R_t = q_t C$ (Sequence of Events below illustrates that within an epoch, receipt of inventory happens after ordering of inventory).

Demand at store in epoch $t$ is denoted by random variable $D_t$.

\section{Constrained Dynamic Program}
The {\em State} in epoch $t$ is defined by the vector:
 $$[IL_t, OO_{t,1}, \ldots OO_{t,L}]$$
 The {\em Action} in epoch $t$ is the number of casepacks ordered, i.e., $q_t$.
 
The {\em Cost} in epoch $t$ is defined as the backroom inventory upon receipt of inventory at the store, i.e., $\max(0, IL_t + R_t - P)$. We set up the problem as an Average-Cost Dynamic Program with the requirements (constraints) that post-demand on-hand inventory $\max(0, IL_t + R_t - D_t) \geq \alpha P$ with probability $\geq \beta$ for all epochs $t$.


\section{Sequence of events in an epoch}

\begin{enumerate}
\item Observe {\em State} (observation of the inventory level $IL_t$  and of the on-orders $OO_{t,1}, \ldots, OO_{t,L}$).
\item Perform {\em Action} (ordering of inventory as number of casepacks $q_t$).
\item Receipt of inventory $R_t$ at the store.
\item Calculate {\em Cost} as the backroom inventory, i.e., $\max(0, IL_t + R_t - P)$.
\item Occurrence of demand at the store (including missed sales, i.e., stockouts at the store).
\item Check if shelf inventory is above requisite threshold, i.e., check if $\max(0, IL_t + R_t - D_t) \geq \alpha P$.
\end{enumerate}

\section{Equations defining Inventory Flow}
The following equations define the inventory flow in any epoch $t$:
$$R_t =
\begin{cases}
OO_{t,1} & \text{ if } L > 0\\
q_t C & \text{ if } L = 0\\
\end{cases}
\mbox{ for all } t$$
$$IL_{t+1} = \max(0, IL_{t} + R_t - D_t) \mbox{ for all } t$$
$$OO_{t+1,k} = OO_{t,k+1} \mbox{ for all } t, \mbox{ for all } 1 \leq k < L$$
$$OO_{t+1,L} = q_t C  \mbox{ for all } t$$


\end{document}