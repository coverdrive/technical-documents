
\documentclass[12pt]{amsart}
\usepackage{geometry} % see geometry.pdf on how to lay out the page. There's lots.
\geometry{a4paper} % or letter or a5paper or ... etc
% \geometry{landscape} % rotated page geometry

% See the ``Article customise'' template for come common customisations

\title{Constrained Dynamic Program for Backroom Minimization ensuring adequate Shelf Inventory}
\author{}
\date{} % delete this line to display the current date

%%% BEGIN DOCUMENT
\begin{document}

\maketitle

\section{Introduction}

$\mathbb{Z}$ refers to the set of integers, $\mathbb{R}$ refers to the set of real numbers. We will subscript $\mathbb{Z}$ and $\mathbb{R}$ to denote appropriate subsets of $\mathbb{Z}$ and $\mathbb{R}$. 
 
We consider a single store and single item served inventory from a supplier with infinite inventory and lead time of $L \in \mathbb{Z}_{\geq 0}$ epochs. Review period is assumed to be 1 epoch. The  customer demand (as units of the item) experienced at the store in epoch $t$ is denoted by random variable $D_t \in \mathbb{Z}_{\geq 0}$.There is a fixed capacity of $P \in \mathbb{Z}_{> 0}$ units for the item on the shelf (planogram) at the store. The item can only be replenished in multiples of $C \in \mathbb{Z}_{> 0}$ units ($C$ refers to the casepack size). Our goal is to identify an ordering policy that minimizes the ``average backroom inventory'' (backroom inventory refers to the store inventory that is in excess of $P$) while ensuring that the shelf inventory in every epoch is at least a specified fraction $\alpha \in [0,1]$ of $P$ with probability at least $\beta \in [0,1]$ (PresMin condition).

\section{Inventory}
\begin{itemize}
\item Denote on-hand inventory (a.k.a. Inventory Level) at the store at the start of epoch $t$ as: $IL_t \in \mathbb{Z}$ (note: $IL_t$ is allowed to go negative if demand is unmet at the store, leading to back-ordering).
\item Denote on-order inventory at the start of epoch $t$ to arrive in $k$ epochs ($1 \leq k \leq L$)  as $OO_{t,k} \in \mathbb{Z}_{\geq 0}$
\end{itemize}

\section{Inventory Movements}
Denote number of casepacks of inventory ordered in epoch $t$ as $q_t \in \mathbb{Z}_{\geq 0}$. The store will receive that inventory of $q_t  C$ in epoch $t + L$. Denote $R_t \in \mathbb{Z}_{\geq 0}$ as the inventory received in epoch $t$. Following the epoch $t$ of inventory ordering and until the epoch $t+L$ of inventory receipt, this quantity $q_t C$ will appear in the flow equations (see below) as on-order $OO_{t+j,L-j+1}, 1 \leq j \leq L$. For the special case where $L = 0$, $R_t = q_t C$ (Sequence of Events below illustrates that within an epoch, inventory receipt is after inventory ordering).



\section{Constrained Dynamic Program}
The {\em State} in epoch $t$ is defined by the vector:
 $$[IL_t, OO_{t,1}, \ldots OO_{t,L}]$$
 The {\em Action} in epoch $t$ is the number of casepacks ordered, i.e., $q_t$.
 
The {\em Cost} in epoch $t$ is defined as the backroom inventory upon receipt of inventory at the store, i.e., $\max(0, IL_t + R_t - P)$. We set up the problem as an Average-Cost Dynamic Program with the requirements (constraints) that on-hand inventory $IL_t \geq \alpha P$ with probability $\geq \beta$ for all epochs $t$.


\section{Sequence of events in an epoch}

\begin{enumerate}
\item Observe {\em State} (observation of the inventory level $IL_t$  and of the on-orders $OO_{t,1}, \ldots, OO_{t,L}$).
\item Check if $IL_t \geq \alpha P$.
\item Perform {\em Action} (ordering of inventory as number of casepacks $q_t$).
\item Receipt of inventory $R_t$ at the store.
\item Calculate {\em Cost} as the backroom inventory, i.e., $\max(0, IL_t + R_t - P)$.
\item Occurrence of demand $D_t$ at the store (including missed sales, i.e., stockouts at the store).
\end{enumerate}

\section{Equations defining Inventory Flow}
The following equations define the inventory flow in any epoch $t$:
$$R_t =
\begin{cases}
OO_{t,1} & \text{ if } L > 0\\
q_t C & \text{ if } L = 0\\
\end{cases}
\mbox{ for all } t$$
$$IL_{t+1} = \max(0, IL_{t} + R_t - D_t) \mbox{ for all } t$$
$$OO_{t+1,k} = OO_{t,k+1} \mbox{ for all } t, \mbox{ for all } 1 \leq k < L$$
$$OO_{t+1,L} = q_t C  \mbox{ for all } t$$

\section{A Heuristic Policy}
We want to order ``enough'' in epoch $t$ so that epoch $t+L+1$ on-hand inventory $IL_{t+L+1} \geq \alpha P$ with probability $\geq \beta$. If we assume backordering, then this can be written as:
$$Pr[IL_t + \sum_{k=1}^L OO_{t,k} + q_t C - \sum_{k=0}^L D_{t+k} \geq \alpha P] \geq \beta$$.

If we also assume that ordering can be in continuous quantities (rather than casepacks), then ordering ``just enough'' in epoch $t$ to satisfy the constraint $Pr[IL_{t+L+1} \geq \alpha P] \geq \beta$ would automatically minimize expected backroom inventory in epoch $t+L$. This ``just enough'' order quantity is given by:
$$\max(0, \alpha P + F_{t,L}^{-1}(\beta) - (IL_t + \sum_{k=1}^L OO_{t,k}))$$
where $F_{t,L}(\cdot)$ is the inverse cumulative mass function of the discrete random variable $\sum_{k=0}^L D_{t+k}$

So we could start with this heuristic as an approximation to the optimal policy for our case of ``no-backordering'' and ``ordering in casepacks'' as follows:

$$q_t = \max(0, \lceil \frac {\alpha P + F_{t,L}^{-1}(\beta) - (IL_t + \sum_{k=1}^L OO_{t,k})} C \rceil)$$

\section{Continuous-Time, Continuous-Inventory, Deterministic Demand}
Now we consider a simple setting where demand happens at a constant deterministic rate in continuous time and in continuous quantity (call the demand rate $\lambda \in \mathbb{R}$ units per time). In this setting, we can adapt the lead time, POG and casepack to be continuous, i.e. $L, P, C \in \mathbb{R}$. 

In this setting, the inventory level at the store follows an EOQ-like sawtooth function with the height of the sawtooth equal to $C$ and the width of the sawtooth equal to $\frac C \lambda$. Our task is to determine the bottom of the sawtooth (denoted $B$) so that the fraction of time the inventory level $IL_t$ goes below $\alpha P$ is no more than $1-\beta$ (adaption of the PresMin condition). Once we establish the bottom of the sawtooth $B$, we will immediately have the top of the sawtooth (denoted $T$) as $B + C$, and consequently, we will have the backroom units per time, and the Order Point.

Due to the constant rate of demand, ensuring that the fraction of time the inventory level $IL_t$ goes below $\alpha P$ is no more than $1-\beta$ is equivalent to ensuring that inventory consumed in a cycle when below the level $\alpha P$ is no more than $(1-\beta)C$ (because inventory consumed in the entire cycle is $C$). So, the PresMin condition can be written as: $B \geq \alpha P - (1-\beta)C$.

We consider two cases:
\begin{itemize}
\item $C \leq (1-\alpha)P + (1-\beta)C$ meaning the casepack is small enough to avoid backroom: if we set $T = P$ (hence, no backroom), $B = P - C \geq \alpha P - (1-\beta)C$ (PresMin condition satisfied)
\item $C > (1-\alpha)P + (1-\beta)C$ meaning the casepack is not small enough to avoid backroom: if we set $B =  \alpha P - (1-\beta)C$, $T = C + \alpha P - (1- \beta)C = \alpha P + \beta C > P$ which means we will have some backroom inventory.
\end{itemize}

Combining the two cases, we can succinctly write $B$ and $T$ as follows:

$$B = \max(P - C, \alpha P - (1-\beta)C)$$
$$T = \max(P, \alpha P + \beta C)$$

The maximum backroom inventory
$$E =T - P = \max(0, \beta C  - (1-\alpha)P)$$

In a given cycle, the time for which there is backroom inventory is $\frac E \lambda$ and so, the aggregate backroom units in any cycle is: $$\frac 1 2 \cdot E \cdot \frac E \lambda$$

Since, each cycle is of duration $\frac C \lambda$, the backroom units per time is:
$$\frac {\frac 1 2 \cdot E \cdot \frac E \lambda} {\frac C \lambda} = \frac {E^2} {2C} = \frac {(\max(0, \beta C  - (1-\alpha)P))^2} {2C}$$

Since the lead time is $L$, we need to place the order exactly $L$ units of time before the inventory level $IL_t = B$. Given the cyclic and regular nature of the sawtooth, we can simply consider the nearest ordering time point before $IL_t = B$. The amount of time between this ordering time point and the time when $IL_t = B$ is:
$$L - \lfloor \frac {L \lambda} C \rfloor \cdot \frac C \lambda$$
and hence, the inventory level at the time of ordering (denoted as Inventory Level Order Point $ILOP$) is:
$$B + \lambda (L - \lfloor \frac {L \lambda} C \rfloor \cdot \frac C \lambda) = \max(P - C, \alpha P - (1- \beta)C) + \lambda (L - \lfloor \frac {L \lambda} C \rfloor \cdot \frac C \lambda)$$
At the time of ordering, the number of casepacks on-order is $\lfloor \frac {L \lambda} C \rfloor$ and so, the number of units of inventory on-order is: $\lfloor \frac {L \lambda} C \rfloor \cdot C$.
So, we can say that at the time of ordering, the inventory position $IP_t$ (defined as $IL_t$ plus the number of units of inventory on-order) is: $$\max(P - C, \alpha P - (1- \beta)C) + \lambda (L - \lfloor \frac {L \lambda} C \rfloor \cdot \frac C \lambda) + \lfloor \frac {L \lambda} C \rfloor \cdot C$$
We can refer to the above expression as the Inventory Position Order Point $IPOP$.



\end{document}