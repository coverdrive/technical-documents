%%%%%%%%%%%%%%%%%%%%%%%%%%%%%%%%%%%%%%%%%
% Beamer Presentation
% LaTeX Template
% Version 1.0 (10/11/12)
%
% This template has been downloaded from:
% http://www.LaTeXTemplates.com
%
% License:
% CC BY-NC-SA 3.0 (http://creativecommons.org/licenses/by-nc-sa/3.0/)
%
%%%%%%%%%%%%%%%%%%%%%%%%%%%%%%%%%%%%%%%%%

%----------------------------------------------------------------------------------------
%	PACKAGES AND THEMES
%----------------------------------------------------------------------------------------

\documentclass[handout]{beamer}

\mode<presentation> {

% The Beamer class comes with a number of default slide themes
% which change the colors and layouts of slides. Below this is a list
% of all the themes, uncomment each in turn to see what they look like.

%\usetheme{default}
%\usetheme{AnnArbor}
%\usetheme{Antibes}
%\usetheme{Bergen}
%\usetheme{Berkeley}
%\usetheme{Berlin}
%\usetheme{Boadilla}
%\usetheme{CambridgeUS}
%\usetheme{Copenhagen}
%\usetheme{Darmstadt}
%\usetheme{Dresden}
%\usetheme{Frankfurt}
%\usetheme{Goettingen}
%\usetheme{Hannover}
%\usetheme{Ilmenau}
%\usetheme{JuanLesPins}
%\usetheme{Luebeck}
\usetheme{Madrid}
%\usetheme{Malmoe}
%\usetheme{Marburg}
%\usetheme{Montpellier}
%\usetheme{PaloAlto}
%\usetheme{Pittsburgh}
%\usetheme{Rochester}
%\usetheme{Singapore}
%\usetheme{Szeged}
%\usetheme{Warsaw}

% As well as themes, the Beamer class has a number of color themes
% for any slide theme. Uncomment each of these in turn to see how it
% changes the colors of your current slide theme.

%\usecolortheme{albatross}
%\usecolortheme{beaver}
%\usecolortheme{beetle}
%\usecolortheme{crane}
%\usecolortheme{dolphin}
%\usecolortheme{dove}
%\usecolortheme{fly}
%\usecolortheme{lily}
%\usecolortheme{orchid}
%\usecolortheme{rose}
%\usecolortheme{seagull}
%\usecolortheme{seahorse}
%\usecolortheme{whale}
%\usecolortheme{wolverine}

%\setbeamertemplate{footline} % To remove the footer line in all slides uncomment this line
%\setbeamertemplate{footline}[page number] % To replace the footer line in all slides with a simple slide count uncomment this line

%\setbeamertemplate{navigation symbols}{} % To remove the navigation symbols from the bottom of all slides uncomment this line
}

\usepackage{graphicx} % Allows including images
\usepackage{booktabs} % Allows the use of \toprule, \midrule and \bottomrule in tables
\usepackage{cool}
\usepackage{amsmath}


%----------------------------------------------------------------------------------------
%	TITLE PAGE
%----------------------------------------------------------------------------------------

\title[ML for Forecasting and Replenishment]{Machine Learning Applications for \\ Demand Forecasting and Inventory Replenishment} % The short title appears at the bottom of every slide, the full title is only on the title page

\author{Ashwin Rao} % Your name
\institute[Wayfair/Stanford] % Your institution as it will appear on the bottom of every slide, may be shorthand to save space
{
Chief Science Officer @ Wayfair and Adjunct Professor @ Stanford University
}

\date{\today} % Date, can be changed to a custom date

\begin{document}
\begin{frame}
\titlepage % Print the title page as the first slide
\end{frame}

\tableofcontents

\section{Machine Learning Overview}

\begin{frame}
\frametitle{Machine Learning}
\includegraphics[width=10cm, height=8cm]{../finance/cme241/MLBranches.PNG}
\end{frame}

\begin{frame}
\frametitle{Machine Learning Overview}
\pause
\begin{itemize}[<+->]
\item Supervised Learning: Predictions based on learning data implications
\item Unsupervised Learning: Unearthing patterns \& structures within data
\item Reinforcement Learning: Optimal Decisioning under Uncertainty
\item ML has been a breakthrough for images, text, games
\item Largely due to the recent success of Deep Neural Networks
\item We are now adapting Deep Learning techniques to other domains
\item Works well when we have plenty of data and plenty of compute
\item ML practice tends to be quite laborious on Big Data Engineering
\item I'm seeing promising results in Finance and Retail applications
\item I cover this in depth in my \href{https://stanford.edu/~ashlearn/RLForFinanceBook/book.pdf}{\underline{\textcolor{blue}{new book}}}
\end{itemize}
\end{frame}


\section{Supply-Chain Headaches}

\begin{frame}
\frametitle{Supply-Chain Headaches and Machine Learning Pills}
\pause
\begin{itemize}[<+->]
\item Customer Demand is hard to forecast
\item Lately, Supply constraints pose challenges too
\item Often, Space and Throughput constraints
\item Network Planning, Labor Planning, Transportation - all challenging
\item Tracking Inventory Count/Location is hard
\item Not easy to control damage, theft, misplacement, spoilage
\item Nirvana: Inventory at right time/place, in right quantity, at low cost
\item Blend of Traditional OR + Modern ML paves the path to Nirvana
\item Today we look at two game-changing applications of ML
\item Demand Forecasting and Inventory Replenishment
\end{itemize}
\end{frame}

\section{Demand Forecasting: Transformers and Embeddings to the rescue}

\begin{frame}
\frametitle{Demand Forecasting}
\pause
\begin{itemize}[<+->]
\item Why do we care about Demand Forecasting?
\item Because it is foundational to LOTs of decisions in Supply-Chain
\item Design/Planning Decisions and Operational/Control Decisions
\item Design/Planning examples: Network, Labor, Transportation
\item Operational/Control examples: Replenishment, Logistics
\item Core: Forecast demand for [SKU group, locations group, time range]
\item Choose your slice thickness in each of these 3 dimensions
\item Thin Slice (local effects): Little data, highly uncertain forecasts
\item Thick Slice (macro effects): Dependent on many external factors
\item Hierarchical forecasting blends macro effects with local effects
\item Good forecast captures structural patterns and regime shifts
\end{itemize}
\end{frame}

\begin{frame}
\frametitle{Similar/Substitutable Products identified with {\em Embeddings}}
\pause
\begin{itemize}[<+->]
\item Demand Forecasting for similar/substitutable set of Products
\item Can we automate identification of similarity/substitutability?
\item Similarity by Visuals or by Descriptions or by Customer Interest
\item Throw this heterogeneous data into deep neural network learning
\item Deep inside the neural network, we find encodings of these products
\item Similar/Substitutable products have similar encodings
\item The technical term for these encodings is {\em Embeddings}
\item Embeddings are low-dimensional numerical representations ({\em Vectors})
\item Capturing the most important features of products
\item Captures various relationships between products, eg: complementarity
\item Based on product images, descriptions, customer interest
\item Embedding vectors have powerful algebraic/geometric properties
\item Embeddings can be used as ML features for transfer learning
\end{itemize}
\end{frame}

\begin{frame}
\frametitle{Book Embeddings flattened to 2-D Vectors}
\includegraphics[width=12cm, height=8cm]{BookEmbeddings.png}
\end{frame}

\begin{frame}
\frametitle{Transformers for learning the old and the new}
\pause
\begin{itemize}[<+->]
\item Forecasting needs to be ``Predictive'' as well as ``Reactive''
\item {\em Predictive:} Capture temporal patterns from long history
\item {\em Reactive:} Pay attention to recent shifts and trends
\item Transformer Networks eating the lunch of good old time-series stats
\item The core methodology is a technique called {\em Self-Attention}
\item It automatically identifies the relative importance of old versus new
\item Transformers are accurate and fast (highly parallelizable)
\item Have found great success in Vision and Text applications
\item Now being ported to numerical time-series applications
\item Embeddings blended with Transformers $\Rightarrow$ Heady Cocktail
\item Not easy to pull off, requires considerable experimentation/tuning
\end{itemize}
\end{frame}

\begin{frame}
\frametitle{Capturing Stock Price Cycles/Trends with Transformers}
\includegraphics[width=11cm, height=7cm]{StockPriceTransformer.jpeg}
\end{frame}

\section{Inventory Replenishment: Reinforcement Learning to the rescue}


\begin{frame}
\frametitle{Inventory Replenishment}
\pause
\begin{itemize}[<+->]
\item Let's consider a simple example of a single SKU in a store
\item The store experiences uncertain daily customer demand
\item The store can order daily from a supplier carrying infinite inventory
\item Cost associated with ordering, and order arrives in a few days
\item Inventory in the store incurs a daily {\em Holding Cost} (per unit)
\item Out-of-Stock incurs a {\em Stockout Cost} (per unit)
\item There's a notion of {\em Ordering Cost} (Labor/Transport)
\item When should one order? And in what quantity?
\item Traditional OR literature focused on closed-form ``solutions''
\item Approx closed-form solutions are often used in real businesses
\item In spite of the limitations of missing out on key real-world features
\end{itemize}
\end{frame}


\begin{frame}
\frametitle{Welcome to the Real-World}
\pause
\begin{itemize}[<+->]
\item Real-world not so simple - frictions, constraints, uncertainties
\item Supply can be constrained/uncertain
\item Space and Throughput constraints
\item Pricing/Marketing has a big influence on demand
\item New products, substitutable products complicate matters
\item In physical stores, there are {\em Presentation-Minimum} requirements
\item Often, products are shipped in casepacks
\item Irregular/Uncertain order frequency and lead times
\item Supply-Chain network might be {\em Multi-Echelon}
\item Inventory Count/Location is often uncertain
\item Cost of Damage, Theft, Misplacement, Spoilage
\item End-of-Season/Obsolescence/Spoilage requires Clearance/Salvage
\item All of this calls for a formal Stochastic Control framework
\end{itemize}
\end{frame}


\begin{frame}
\frametitle{The Markov Decision Process Framework}
\includegraphics[width=12cm, height=7cm]{../finance/cme241/MDP.png}
\end{frame}

\begin{frame}
\frametitle{Components of the MDP Framework}
\pause
\begin{itemize}[<+->]
\item The {\em Agent} and the {\em Environment} interact in a time-sequenced loop
\item {\em Agent} responds to [{\em State}, {\em Reward}] by taking an {\em Action}
\item {\em Environment} responds by producing next step's (random) {\em State}
\item {\em Environment} also produces a (random) number we call {\em Reward}
\item Goal of {\em Agent} is to maximize {\em Expected Sum} of all future {\em Reward}s
\item By controlling the ({\em Policy} : {\em State} $\rightarrow$ {\em Action}) function
\item This is a dynamic (time-sequenced control) system under uncertainty
\item MDP framework enables modeling real-world Replenishment problems
\end{itemize}
\end{frame}

\begin{frame}
\frametitle{How a baby learns to walk}
\includegraphics[width=13cm, height=8cm]{../finance/cme241/BabyMDP.jpg}
\end{frame}

\begin{frame}
\frametitle{Many real-world problems fit this MDP framework}
\pause
\begin{itemize}[<+->]
\item Self-driving vehicle (speed/steering to optimize safety/time)
\item Game of Chess (Boolean {\em Reward} at end of game)
\item Inventory Replenishment to ensure high availability at low cost
\item Make a humanoid robot walk/run on difficult terrains
\item Manage an investment portfolio
\item Control a power station
\item Optimal decisions during a football game
\item Strategy to win an election (high-complexity MDP)
\end{itemize}
\end{frame}

\begin{frame}
\frametitle{Self-Driving Vehicle}
\includegraphics[width=13cm, height=8cm]{../finance/cme241/CarMDP.jpg}
\end{frame}

\begin{frame}
\frametitle{Real-World Replenishment as a Markov Decision Process}
\pause
\begin{itemize}[<+->]
\item MDP {\em State} is current Inventory Level at the store
\item {\em State} also includes current in-transit inventory
\item {\em Action} is the multiple of casepack to order (or not order)
\item {\em Reward} function involves all of the costs we went over
\item State transitions governed by all of the uncertainties we went over
\item Solve: Dynamic Programming or Reinforcement Learning
\item Curse of Dimensionality and Curse of Modeling $\Rightarrow$ RL
\end{itemize}
\end{frame}


\begin{frame}
\frametitle{How RL Works: Learning from Samples of Data}
\pause
\begin{itemize}[<+->]
\item RL incrementally learns from state/reward transitions data
\item Typically served by a simulator acting as a {\em Simulated Environment}
\item RL is a ``trial-and-error'' approach linking {\em Actions} to {\em Rewards}
\item Try different actions \& learn what works, what doesn't
\item Deep Neural Networks are typically used for function approximation
\item Big Picture: Sampling and Function Approximation come together
\item RL algorithms are clever about balancing  ``explore'' versus ``exploit''
\item Promise of modern A.I. is based on success of RL algorithms
\item Potential for automated decision-making in many industries
\item In 10-20 years: Bots that act or behave more optimal than humans
\item RL already solves various low-complexity real-world problems
\item RL has many applications in Suppy-Chain
\end{itemize}
\end{frame}

\end{document}
