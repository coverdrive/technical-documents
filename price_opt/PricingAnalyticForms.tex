   
\documentclass[12pt]{amsart}
   \usepackage{geometry} % see geometry.pdf on how to lay out the page. There's lots.
   \geometry{a4paper} % or letter or a5paper or ... etc
   \usepackage{amsthm}
   \usepackage{float}
   \usepackage{graphicx}
   \newtheorem{prop}{Proposition}
   % \geometry{landscape} % rotated page geometry
   
   % See the ``Article customise'' template for come common customisations
   
   \title{Analytic Forms for Clearance Price Optimization}
   \author{Ashwin Rao}
   \date{} % delete this line to display the current date
   
   %%% BEGIN DOCUMENT
   \begin{document}
   
   \maketitle 
   \section{Introduction}
   This is a brief article to derive analytic forms for Clearance Price Optimization.    
   \section{Bellman Equation}
    The {\em State} at time $t \in \{0, 1, \ldots, T\}$ is the Inventory $I_t \in \mathbb{Z}_{\geq 0}$. The {\em Action} at time $t$ is the Price $p_t \in \mathbb{R}_{\geq 0}$ set at time $t$. Assume the probability mass function (PMF) of demand $k_t \in \mathbb{Z}_{\geq 0}$ at time $t$ has mean $f(p_t)$ and variance $g(p_t)$ for some given functions $f, g : \mathbb{R}_{\geq 0} \rightarrow \mathbb{R}_{\geq 0}$. The {\em Reward} at time $t$ is $p_t \cdot \min(k_t, I_t)$. Then, for all $t = 0, 1, \ldots, T-1$, we have the following Bellman Equation:
    
    $$V_t^*(I_t) = \max_{p_t} \{ \sum_{k_t=0}^{\infty} Pr[k_t; f(p_t), g(p_t)] \cdot (p_t \cdot \min(k_t, I_t) + V_{t+1}^*(\max(I_t - k_t, 0))) \}$$
    $$V_t^*(0) = 0, V_T^*(I_T) = 0$$
   
   As an example, consider the Poisson distribution for the PMF of demand given by Poisson mean $f(p_t) = \alpha \cdot e^{-\beta \cdot p_t}$ for all $t = 0, 1, \ldots, T-1$. Then,
   
    $$V_t^*(I_t) = \max_{p_t} \{ \sum_{k_t=0}^{I_t - 1} \frac {e^{-\alpha \cdot e^{-\beta \cdot p_t}} \cdot \alpha^{k_t} \cdot e^{-\beta \cdot k_t \cdot p_t}} {k_t!} \cdot (p_t \cdot k_t + V_{t+1}^*(I_t - k_t)) +  \sum_{k_t=I_t}^{\infty} \frac {e^{-\alpha \cdot e^{-\beta \cdot p_t}} \cdot \alpha^{k_t} e^{-\beta \cdot k_t \cdot p_t}} {k_t!} \cdot p_t \cdot I_t \}$$
    $$V_t^*(0) = 0, V_T^*(I_T) = 0$$
   
   For $t=T-1$, we take the derivative of the right hand side with respect to $p_{T-1}$ and set to 0. This enables us to solve for $p_{T-1}^*$ and $V_{T-1}^*$. Then we do the same for $t=T-2$ and proceed back in time to obtain $p_t^*$ and $V_t^*$ for all $t = 0, 1, \ldots T-1$.
   
   The above technique is easy for the case of deterministic demand.
   
   \section{Deterministic Case}
   Here we consider the case where demand is a deterministic function of price, denoted as $f : \mathbb{R}_{\geq 0} \rightarrow \mathbb{R}_{\geq 0}$ for each time step $t = 0, 1, \ldots T-1$. For the deterministic case, since demand at each time step $t$ is a continuous variable ($\in \mathbb{R}_{\geq 0}$), the inventory $I_t$ at time $t$ is a continuous variable ($\in \mathbb{R}_{\geq 0}$). Then, for all $t = 0, 1, \ldots T - 1$, we have the Bellman Equation:
   
   $$V_t^*(I_t) = \max_{p_t} \{ p_t \cdot \min(f(p_t), I_t) + V_{t+1}^*(\max(I_t - f(p_t), 0)) \}$$
   $$V_t^*(0) = 0,  V_T^*(I_T) = 0$$
  
  \subsection{Simplifying the Bellman Equation}
  We will assume that $f$ has an inverse $f^{-1} : \mathbb{R}_{\geq 0} \rightarrow \mathbb{R}_{\geq 0}$ and that $f$ is a monotonically decreasing function. Then, we have the following Proposition.
  \begin{prop}
  Optimal Pricing will not permit the demand at time $t$ to exceed the Inventory $I_t$ at time $t$, for all $t = 0, 1, \ldots, T-1$, i.e., $f(p_t^*) \leq I_t$ or equivalently, $p_t^* \geq f^{-1}(I_t)$.
  \end{prop}
  \begin{proof}
  Assume the contrary, that $f(p_t^*) > I_t$, or equivalently, $p_t^* < f^{-1}(I_t)$. Since demand $f(p_t^*)$ exceeds $I_t$, $I_{t+1} = 0$ and so, $V_{t+1}^*(I_{t+1}) = 0$. Therefore, $V_t^*(I_t) = p_t^* \cdot I_t < f^{-1}(I_t) \cdot I_t$. This says that a price of $f^{-1}(I_t)$ producing a demand of $I_t$ attains a {\em Reward} of  $f^{-1}(I_t) \cdot I_t$ at time step $t$ that exceeds the assumed optimal value function $V_t^*(I_t) \Rightarrow $ Reductio Ad Absurdum.
  \end{proof}
  
  An important consequence of this Proposition is that we can simplify the Bellman Equation:
  
  $$V_t^*(I_t) = \max_{p_t \geq f^{-1}(I_t)} \{ p_t \cdot f(p_t) + V_{t+1}^*(I_t - f(p_t)) \}$$
   $$V_t^*(0) = 0,  V_T^*(I_T) = 0$$

   \subsection{Exponential Price Function}
   Assume $f(p_t) = \alpha \cdot e^{-\beta \cdot p_t}$ for some given $\alpha, \beta \in \mathbb{R}_{> 0}$. Then,
   
   $$V_t^*(I_t) = \max_{p_t \geq \frac 1 \beta \log \frac \alpha {I_t}} \{ p_t \cdot \alpha \cdot e^{-\beta \cdot p_t}  + V_{t+1}^*(I_t - \alpha \cdot e^{-\beta \cdot p_t}) \}$$
   $$V_t^*(0) = 0, V_T^*(I_T) = 0$$
   
   We note that as $I_t \to 0$, $p_t^* \geq \frac 1 \beta \log \frac \alpha {I_t} \to \infty$, Demand under Optimal Price ($= \alpha \cdot e^{-\beta \cdot p_t^*}$) $\to 0$, and {\em Reward} under Optimal Price ($= p_t^* \cdot \alpha \cdot e^{-\beta \cdot p_t^*}$) $\to 0$.
   
   \subsection{Backward Recursion: Optimal Price/Value Function for $t = T-1$} 
   We start with time $t = T-1$ and solve the Bellman Equation by recursing back in time. The Bellman equation for time step $t = T-1$ is:
      
   $$V_{T-1}^*(I_{T-1}) = \max_{p_{T-1} \geq \frac 1 \beta \log \frac \alpha {I_{T-1}}} \{ \alpha \cdot p_{T-1} \cdot  e^{-\beta \cdot p_{T-1}} \}$$
   
   In order to perform the constrained maximization above, let us examine the function $h(x) = \alpha \cdot x \cdot e^{-\beta x}$ for all $x \in \mathbb{R}_{\geq 0}$.
   \begin{itemize}
   \item $h(0) = 0$
   \item $\lim_{x \to \infty} h(x) = 0$
   \item $h'(x) = \alpha \cdot e^{-\beta x} \cdot (1 - \beta x)$ and so, $h'(x) > 0$ for $x < \frac 1 \beta, h'(x) = 0$ for $x = \frac 1 \beta$, $h'(x) < 0$ for $x > \frac 1 \beta$
   % \item $h''(x) = \alpha \cdot \beta \cdot e^{-\beta x} \cdot (\beta x - 2)$ and so, $h''(x) < 0$ for $x < \frac 2 \beta, h''(x) = 0$ for $x = \frac 2 \beta$, $h''(x) < 0$ for $x > \frac 2 \beta$
   \end{itemize}

From the above analysis, we note that $h(x)$ has a single peak at $x = \frac 1 \beta$. If this peak occurs within the above-specified range for the price ($p_{T-1} \geq \frac 1 \beta \log \frac \alpha {I_{T-1}}$), then the optimal price is where the peak occurs ($= \frac 1 \beta$). Otherwise, the optimal price is the lower bound of this price range ($= \frac 1 \beta \log \frac \alpha {I_{T-1}}$). We also note that the above condition ($\frac 1 \beta \geq \frac 1 \beta \log \frac \alpha {I_{T-1}}$) can be succinctly expressed as: $I_t \geq \frac \alpha e$. Below, we summarize the Optimal Price and consequent Optimal Value Function for time step $t = T-1$:

$$
p_{T-1}^*(I_{T-1}) = 
\begin{cases}
\frac 1 \beta & \text{if } I_{T-1} \geq \frac \alpha e \\
\frac 1 \beta \log{ \frac \alpha {I_{T-1}}} & \text{if } I_{T-1} < \frac \alpha e\\
\end{cases}
$$

$$
V_{T-1}^*(I_{T-1}) = 
\begin{cases}
\frac {\alpha} {\beta \cdot e} & \text{if } I_{T-1} \geq \frac \alpha e \\
\frac {I_{T-1}} {\beta} \log{ \frac {\alpha} {I_{T-1}}} & \text{if } I_{T-1} < \frac \alpha e\\
\end{cases}
$$

\subsection{Backward Recursion: Optimal Price/Value Function for $t = T-2$} 
   
Stepping back in time, the Bellman Equation for time step $t=T-2$ is:
$$V_{T-2}^*(I_{T-2}) = \max_{p_{T-2} \geq \frac 1 \beta \log \frac \alpha {I_{T-2}}} \{ p_{T-2} \cdot \alpha \cdot e^{-\beta \cdot p_{T-2}}  + V_{T-1}^*(I_{T-2} - \alpha \cdot e^{-\beta \cdot p_{T-2}}) \}$$
 
We start by making a couple of key observations:
\begin{itemize}
\item $V_{T-1}^*(I_{T-1})$ is a monotonically increasing function of $I_{T-1}$ for $I_{T-1} < \frac \alpha e$ and is constant (= $\frac 1 \beta$) for $I_{T-1} \geq \frac \alpha e$
\item $I_{T-1}(p_{T-2}) = I_{T-2} - \alpha \cdot e^{-\beta \cdot p_{T-2}}$ is a monotonically increasing function of $p_{T-2}$ for the entire range of $I_{T-1}$ from $0$ to $I_{T-2}$ (see equations below):
$$I_{T-1}(p_{T-2} = \frac 1 \beta \log \frac \alpha {I_{T-2}}) = 0$$
$$I_{T-1}(p_{T-2} = \frac 1 \beta \log \frac \alpha {I_{T-2} - \frac \alpha e}) = \frac \alpha e$$
$$\lim_{p_{T-2} \to \infty} I_{T-1}(p_{T-2}) = I_{T-2}$$
\end{itemize}

Combining these two observations, we note that $V_{T-1}^*(I_{T-2} - \alpha \cdot e^{-\beta \cdot p_{T-2}})$ is a monotonically increasing function of $p_{T-2}$ for $\frac 1 \beta \log \frac \alpha {I_{T-2}} \leq  p_{T-2} < \frac 1 \beta \log \frac \alpha {I_{T-2} - \frac \alpha e}$ and is constant for $p_{T-2} \geq \frac 1 \beta \log \frac \alpha {I_{T-2} - \frac \alpha e}$. We also know that $p_{T-2} \cdot \alpha \cdot e^{-\beta \cdot p_{T-2}}$ has a single peak at $p_{T-2} = \frac 1 \beta$. 

Therefore, if $\frac 1 \beta \geq \frac 1 \beta \log \frac \alpha {I_{T-2} - \frac \alpha e}$ (or equivalently, $I_{T-2} \geq \frac {2 \alpha} e$), then we can assert that $p_{T-2} \cdot \alpha \cdot e^{-\beta \cdot p_{T-2}}  + V_{T-1}^*(I_{T-2} - \alpha \cdot e^{-\beta \cdot p_{T-2}})$ (right-hand side of Bellman Equation for $t=T-2$) will have a single peak at $p_{T-2} = \frac 1 \beta$.

Summarizing the above arguments, we state that for $I_{T-2} \geq \frac {2 \alpha} e$,

$$p_{T-2}^*(I_{T-2}) = \frac 1 \beta$$
$$V_{T-2}^*(I_{T-2}) = \frac {2 \alpha} {\beta \cdot e}$$

Now we come to the remaining case of $I_{T-2} <  \frac {2 \alpha} e$. This case corresponds to: $\frac 1 \beta < \frac 1 \beta \log \frac \alpha {I_{T-2} - \frac \alpha e}$. This means $p_{T-2} \cdot \alpha \cdot e^{-\beta \cdot p_{T-2}}  + V_{T-1}^*(I_{T-2} - \alpha \cdot e^{-\beta \cdot p_{T-2}})$ (right-hand side of Bellman Equation for $t=T-2$) will peak for $p_{T-2}$ within the range: 
$$\frac 1 \beta \leq p_{T-2} \leq \frac 1 \beta \log \frac \alpha {\max(I_{T-2} - \frac \alpha e, 0)}$$

We also note that this range of $p_{T-2}$ corresponds to $I_{T-1} < \frac \alpha e$, which in turn corresponds to $V_{T-1}^*(I_{T-1}) = \frac {I_{T-1}} {\beta} \log{ \frac {\alpha} {I_{T-1}}}$. Hence, we can formally state the case of $I_{T-2} <  \frac {2 \alpha} e$ as maximization of:

$$s(p_{T-2}) = p_{T-2} \cdot \alpha \cdot e^{-\beta \cdot p_{T-2}}  + \frac {I_{T-2} - \alpha \cdot e^{-\beta \cdot p_{T-2}}} {\beta} \log{ \frac {\alpha} {I_{T-2} - \alpha \cdot e^{-\beta \cdot p_{T-2}}}}$$

under the following range constraints for $p_{T-2}$:

$$\frac 1 \beta \leq p_{T-2} \leq \frac 1 \beta \log \frac \alpha {\max(I_{T-2} - \frac \alpha e, 0)}$$

Let us now analyze the function:

$$s(x) = x \cdot \alpha \cdot e^{-\beta x}  + \frac {I - \alpha \cdot e^{-\beta x}} {\beta} \log \frac {\alpha} {I - \alpha \cdot e^{-\beta x}}$$

for all $x \in \mathbb{R}_{\geq 0}$.

\begin{itemize}
\item $s(0) = \frac {I - \alpha} {\beta} \log \frac {\alpha} {I - \alpha}$
\item $\lim_{x \to \infty} s(x) = \frac I {\beta} \log \frac {\alpha} I$
\item $s'(x) = \alpha \cdot e^{-\beta x} \cdot (\log \frac \alpha {I_{T-2} - \alpha \cdot e^{-\beta x}} - \beta x)$ and so, $h'(x) > 0$ for $x < \frac 1 \beta \log \frac {2 \alpha} {I_{T-2}}, h'(x) = 0$ for $x = \frac 1 \beta \log \frac {2 \alpha} {I_{T-2}}$, $h'(x) < 0$ for $x > \frac 1 \beta \log \frac {2 \alpha} {I_{T-2}}$
\end{itemize}

From the above analysis, we note that $s(x)$ has a single peak at $x = \frac 1 \beta \log \frac {2 \alpha} {I_{T-2}}$. Since we are considering the case of $I_{T-2} < \frac {2 \alpha} e$, the Optimal Price $p_{T-2}^*(I_{T-2}) = \frac 1 \beta \log \frac {2 \alpha} {I_{T-2}}$ satisfies the two constraints $p_{T-2} \geq \frac 1 \beta$ and $p_{T-2} \leq \frac 1 \beta \log \frac \alpha {\max(I_{T-2} - \frac \alpha e, 0)}$. 

Summarizing the above arguments, we state that for $I_{T-2} < \frac {2 \alpha} e$,

$$p_{T-2}^*(I_{T-2}) = \frac 1 \beta \log \frac {2 \alpha} {I_{T-2}}$$
$$V_{T-2}^*(I_{T-2}) = \frac {I_{T-2}} \beta \log \frac {2 \alpha} {I_{T-2}}$$

So now we are ready to summarize the Optimal Price and Optimal Value Function for $t = T-2$:

$$
p_{T-2}^*(I_{T-2}) = 
\begin{cases}
\frac 1 \beta & \text{if } I_{T-2} \geq \frac {2 \alpha} e \\
\frac 1 \beta \log{ \frac {2 \alpha} {I_{T-2}}} & \text{if } I_{T-2} < \frac {2 \alpha} e\\
\end{cases}
$$

$$
V_{T-2}^*(I_{T-2}) = 
\begin{cases}
\frac {2 \alpha} {\beta \cdot e} & \text{if } I_{T-2} \geq \frac {2 \alpha} e \\
\frac {I_{T-2}} {\beta} \log{ \frac {2 \alpha} {I_{T-2}}} & \text{if } I_{T-2} < \frac {2 \alpha} e\\
\end{cases}
$$

\subsection{Backward Recursion: Optimal Price/Value Function for all $t = 0, 1, \ldots T-1$} 
   
Stepping back in a likewise manner gives us the following Optimal Price and Optimal Value Function for all $t = 0, 1, \ldots, T-1$.

$$
p_t^*(I_t) = 
\begin{cases}
\frac 1 \beta & \text{if } I_t \geq \frac {(T-t) \alpha} e \\
\frac 1 \beta \log{ \frac {(T-t) \alpha} {I_t}} & \text{if } I_t < \frac {(T-t) \alpha} e\\
\end{cases}
$$

$$
V_t^*(I_t) = 
\begin{cases}
\frac {(T-t) \alpha} {\beta \cdot e} & \text{if } I_t \geq \frac {(T-t) \alpha} e \\
\frac {I_t} {\beta} \log{ \frac {(T-t) \alpha} {I_t}} & \text{if } I_t < \frac {(T-t) \alpha} e\\
\end{cases}
$$

The interesting aspect of this result is that {\bf there are only two possible optimal prices for the deterministic case} (separated by the line $I_t = \frac {(T-t) \alpha} e$ which we will refer to as the ``Optimal Pricing Frontier'').

\begin{figure}[H]
  \includegraphics[width=0.82\textwidth]{frontier.jpg}
  \caption{Optimal Pricing Frontier}
\end{figure}

We note that:
\begin{itemize}
\item Above the ``Optimal Pricing Frontier'', Demand (under Optimal Pricing) occurs at the rate of $\frac \alpha e$ per time step (i.e., inventory trajectory is a straight line parallel to the ``Optimal Pricing Frontier'')
\item Below the ``Optimal Pricing Frontier'', Demand (under Optimal Pricing) occurs at the rate of $\frac {I_t} {T-t}$ per time step (i.e., inventory trajectory is a straight line taking the inventory to exactly 0 at $t=T$).
\end{itemize}

Hence, the demand rate (under Optimal Pricing) is higher above the ``Optimal Pricing Frontier'', and correspondingly Optimal Price is lower above the ``Optimal Pricing Frontier'' (as expected). 


   
   \section{Calibration to ``Demand Lift''}
   
   Let us assume that the ``base price'' is 1, for which the demand is $D_0$. Let us assume that the demand for ``half price off'' (i.e., for price of 0.5) is $D_{0.5} = (1+L)\cdot D_0$ (we will refer to $L$ as the ``Demand Lift''). Let us calibrate the function $f(p) = \alpha \cdot e^{-\beta \cdot p}$ to these values:
   
$$D_0 = \alpha \cdot e^{-\beta}, D_{0.5} = \alpha \cdot e^{-\frac {\beta} 2}$$
Solving for $\alpha$ and $\beta$:
$$\alpha = \frac {D_{0.5}^2} {D_0} = D_0 \cdot (1+L)^2$$
$$\beta = 2 \log{(\frac {D_{0.5}} {D_0})} = 2 \log{(1 + L)}$$
$$f(p) = D_{0.5}^{2(1-p)} \cdot D_0^{2p-1} = D_0 \cdot (1+L)^{2(1-p)}$$

So, the ``Optimal Price Frontier'' (for the case of Deterministic Demand) is given by the line:
$$I_t = \frac {(T-t) \cdot D_0 \cdot (1+L)^2} {e}$$
Above the ``Optimal Price Frontier'', we have:
$$p_t^*(I_t) = \frac 1 {2 \log{(1+L)}}$$
$$V_t^*(I_t) = \frac {(T-t) \cdot D_0 \cdot (1+L)^2} {2 \cdot e \cdot \log{(1+L)}} $$

Below the ``Optimal Price Frontier'', we have:
$$p_t^*(I_t) = 1 + \frac {\log \frac {(T-t)D_0} {I_t}} {2 \log (1+L)}$$
$$V_t^*(I_t) = I_t(1 + \frac {\log \frac {(T-t)D_0} {I_t}} {2 \log (1+L)}) $$
   
     
   \end{document}