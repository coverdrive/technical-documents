%%%%%%%%%%%%%%%%%%%%%%%%%%%%%%%%%%%%%%%%%
% Beamer Presentation
% LaTeX Template
% Version 1.0 (10/11/12)
%
% This template has been downloaded from:
% http://www.LaTeXTemplates.com
%
% License:
% CC BY-NC-SA 3.0 (http://creativecommons.org/licenses/by-nc-sa/3.0/)
%
%%%%%%%%%%%%%%%%%%%%%%%%%%%%%%%%%%%%%%%%%

%----------------------------------------------------------------------------------------
%	PACKAGES AND THEMES
%----------------------------------------------------------------------------------------

\documentclass{beamer}

\mode<presentation> {

% The Beamer class comes with a number of default slide themes
% which change the colors and layouts of slides. Below this is a list
% of all the themes, uncomment each in turn to see what they look like.

%\usetheme{default}
%\usetheme{AnnArbor}
%\usetheme{Antibes}
%\usetheme{Bergen}
%\usetheme{Berkeley}
%\usetheme{Berlin}
%\usetheme{Boadilla}
%\usetheme{CambridgeUS}
%\usetheme{Copenhagen}
%\usetheme{Darmstadt}
%\usetheme{Dresden}
%\usetheme{Frankfurt}
%\usetheme{Goettingen}
%\usetheme{Hannover}
%\usetheme{Ilmenau}
%\usetheme{JuanLesPins}
%\usetheme{Luebeck}
\usetheme{Madrid}
%\usetheme{Malmoe}
%\usetheme{Marburg}
%\usetheme{Montpellier}
%\usetheme{PaloAlto}
%\usetheme{Pittsburgh}
%\usetheme{Rochester}
%\usetheme{Singapore}
%\usetheme{Szeged}
%\usetheme{Warsaw}

% As well as themes, the Beamer class has a number of color themes
% for any slide theme. Uncomment each of these in turn to see how it
% changes the colors of your current slide theme.

%\usecolortheme{albatross}
%\usecolortheme{beaver}
%\usecolortheme{beetle}
%\usecolortheme{crane}
%\usecolortheme{dolphin}
%\usecolortheme{dove}
%\usecolortheme{fly}
%\usecolortheme{lily}
%\usecolortheme{orchid}
%\usecolortheme{rose}
%\usecolortheme{seagull}
%\usecolortheme{seahorse}
%\usecolortheme{whale}
%\usecolortheme{wolverine}

%\setbeamertemplate{footline} % To remove the footer line in all slides uncomment this line
%\setbeamertemplate{footline}[page number] % To replace the footer line in all slides with a simple slide count uncomment this line

%\setbeamertemplate{navigation symbols}{} % To remove the navigation symbols from the bottom of all slides uncomment this line
}

\usepackage{graphicx} % Allows including images
\usepackage{booktabs} % Allows the use of \toprule, \midrule and \bottomrule in tables
\usepackage{cool}
\usepackage{tikz}
\usepackage{amsmath}
\DeclareMathOperator*{\argmax}{argmax}
\DeclareMathOperator*{\argmin}{argmin}
\usetikzlibrary{positioning}

%----------------------------------------------------------------------------------------
%	TITLE PAGE
%----------------------------------------------------------------------------------------

\title[Stochastic Calculus Foundations]{Refresher on Stochastic Calculus Foundations} % The short title appears at the bottom of every slide, the full title is only on the title page

\author{Ashwin Rao} % Your name
\institute[Stanford] % Your institution as it will appear on the bottom of every slide, may be shorthand to save space
{
ICME, Stanford University
 % Your institution for the title page
}

\date{\today} % Date, can be changed to a custom date

\begin{document}
\begin{frame}
\titlepage % Print the title page as the first slide
\end{frame}

\begin{frame}
\frametitle{Continuous and Non-Differentiable Sample Paths}
\begin{itemize}
\item Sample paths of Brownian motion $z_t$ are continuous
\item Sample paths of $z_t$ are almost always non-differentiable, meaning
$$\lim_{h \rightarrow 0} \frac {z_{t+h} - z_t} {h} \mbox{ is almost always infinite}$$
\item The intuition is that $\frac {dz_t} {dt}$ has standard deviation of $\frac 1 {\sqrt{dt}}$, which goes to $\infty$ as $dt$ goes to 0
\end{itemize}
\end{frame}

\begin{frame}
\frametitle{Infinite Total Variation of Sample Paths}
\begin{itemize}
\item Sample paths of Brownian motion are of infinite total variation, i.e.
$$\lim_{h \rightarrow 0} \sum_{i=m}^{n-1} |z_{(i+1)h} - z_{ih}| \mbox{ is almost always infinite}$$
\item More succinctly, we write
$$\int_S^T |dz_t| = \infty \mbox{ (almost always)}$$
\end{itemize}
\end{frame}

\begin{frame}
\frametitle{Finite Quadratic Variation of Sample Paths}
\begin{itemize}
\item Sample paths of Brownian Motion are of finite quadratic variation, i.e.
$$\lim_{h \rightarrow 0} \sum_{i=m}^{n-1} (z_{(i+1)h} - z_{ih})^2 = h(n-m)$$ 
\item More succinctly, we write
$$\int_S^T (dz_t)^2= T-S$$
\item This means it's expected value is $T-S$ and it's variance is 0
\item This leads to Ito's Lemma (Taylor series with $(dz_t)^2$ replaced with $dt$)
\item This also leads to Ito Isometry (next slide)
\end{itemize}
\end{frame}

\begin{frame}
\frametitle{Ito Isometry}
\begin{itemize}
\item Let $X_t : [0,T] \times \Omega \rightarrow \mathbb{R}$ be a stochastic process adapted to filtration $\mathcal{F}$ of brownian motion $z_t$.
\item Then, we know that the Ito integral $\int_0^T X_t \cdot dz_t$ is a martingale
\item Ito Isometry tells us about the variance of $\int_0^T X_t \cdot dz_t$
$$\mathbb{E}[(\int_0^T X_t \cdot dz_t)^2] = \mathbb{E}[\int_0^T X_t^2 \cdot dt]$$
\item Extending this to two Ito integrals, we have:
$$\mathbb{E}[(\int_0^T X_t \cdot dz_t) (\int_0^T Y_t \cdot dz_t)] = \mathbb{E}[\int_0^T X_t \cdot Y_t  \cdot dt]$$
\end{itemize}
\end{frame}

\begin{frame}
\frametitle{Fokker-Planck equation for PDF of a Stochastic Process}
\begin{itemize}
\item We are given the following stochastic process:
$$dX_t = \mu(X_t, t) \cdot dt + \sigma(X_t, t) \cdot dz_t$$
\item The Fokker-Planck equation of this process is the PDE:
$$\pderiv{p(x,t)}{t} = -\pderiv{\{\mu(x,t) \cdot p(x,t)\}}{x} + \pderiv[2]{\{\frac {\sigma^2(x,t)} 2 \cdot p(x,t)\}}{x}$$
where $p(x,t)$ is the probability density function of $X_t$
\item The Fokker-Planck equation is used for problems where the initial distribution is known (Kolmogorov forward equation)
\item However, if the problem is to know the distribution at previous times, the Feynman-Kac formula can be used (a consequence of the Kolmogorov backward equation)
\end{itemize}
\end{frame}

\begin{frame}
\frametitle{Feynman-Kac Formula (PDE-SDE linkage)}
\begin{itemize}
\item Consider the partial differential equation for $u : \mathbb{R} \times [0,T] \rightarrow \mathbb{R}$:
$$\pderiv{u(x,t)}{t} + \mu(x,t) \pderiv{u(x,t)}{x} + \frac {\sigma^2(x,t)} 2 \pderiv[2]{u(x,t)}{x} - V(x,t) u(x,t) = f(x,t)$$ subject to $u(x,T) = \psi(x)$,
where $\mu, \sigma, V, f, \psi$ are known functions.
\item Then the Feynman-Kac formula tells us that the solution $u(x,t)$ can be written as the following conditional expectation:
$$\mathbb{E}[(\int_t^T e^{-\int_t^u V(X_s, s) ds} \cdot f(X_u, u) \cdot du) + e^{-\int_t^T V(X_u, u) du} \cdot \psi(X_T) | X_t = x]$$
such that $X_u$ is the following Ito process with initial condition $X_t = x$:
$$dX_u = \mu(X_u, u) \cdot du + \sigma(X_u, u) \cdot dz_u$$
\end{itemize}
\end{frame}

\begin{frame}
\frametitle{Stopping Time}
\begin{itemize}
\item Stopping time $\tau$ is a ``random time'' (random variable) interpreted as time at which a given stochastic process exhibits certain behavior
\item Stopping time often defined by a ``stopping policy'' to decide whether to continue/stop a process based on present position and past events
\item Random variable $\tau$ such that $Pr[\tau \leq t]$ is in $\sigma$-algebra $\mathcal{F}_t$, for all $t$
\item Deciding whether $\tau \leq t$ only depends on information up to time $t$
\item Hitting time of a Borel set $A$ for a process $X_t$ is the first time $X_t$ takes a value within the set $A$
\item Hitting time is an example of stopping time. Formally, 
$$T_{X,A} = \min \{t \in \mathbb{R} | X_t \in A\}$$
eg: Hitting time of a process to exceed a certain fixed level
\end{itemize}
\end{frame}

\begin{frame}
\frametitle{Optimal Stopping Problem}
\begin{itemize}
\item Optimal Stopping problem for Stochastic Process $X_t$: 
$$V(x) = \max_{\tau} \mathbb{E}[G(X_{\tau})|X_0 = x]$$
 where $\tau$ is a set of stopping times of $X_t$, $V(\cdot)$ is called the value function, and $G$ is called the reward (or gain) function.
\item Note that sometimes we can have several stopping times that maximize $\mathbb{E}[G(X_{\tau})]$ and we say that the optimal stopping time
is the smallest stopping time achieving the maximum value.
\item Example of Optimal Stopping: Optimal Exercise of American Options
\begin{itemize}
\item $X_t$ is stochastic process for underlying security's price
\item $x$ is underlying security's current price
\item $\tau$ is set of exercise times corresponding to various stopping policies
\item $V(\cdot)$ is American option price as function of underlying's current price
\item $G(\cdot)$ is the option payoff function
\end{itemize}
\end{itemize}
\end{frame}

\begin{frame}
\frametitle{Markov Property}
\begin{itemize}
\item Markov property says that the $\mathcal{F}_t$-conditional PDF of $X_{t+h}$ depends only on the present state $X_t$
\item Strong Markov property says that for every stopping time $\tau$, the $\mathcal{F}_{\tau}$-conditional PDF of $X_{\tau+h}$ depends only on $X_{\tau}$
\end{itemize}
\end{frame}

\begin{frame}
\frametitle{Infinitesimal Generator and Dynkin's Formula}
\begin{itemize}
\item Infinitesimal Generator of a time-homogeneous $\mathbb{R}^n$-valued diffusion $X_t$ is the PDE operator $A$ (operating on functions $f: \mathbb{R}^n \rightarrow \mathbb{R}$) defined as
$$ A \bullet f(x) = \lim_{t \rightarrow 0} \frac {\mathbb{E}[f(X_t) | X_0 = x] - f(x)} {t}$$
\item For $\mathbb{R}^n$-valued diffusion $X_t$ given by: $dX_t = \mu(X_t) \cdot dt + \sigma(X_t) \cdot dz_t$,
$$ A \bullet f(x) = \sum_i \mu_i(x) \pderiv{f}{x_i} (x) + \sum_{i,j} (\sigma(x) \sigma(x)^T)_{i,j} \frac {\partial^2 f} {\partial x_i \, \partial x_j} (x)$$
\item If $\tau$ is stopping time conditional on $X_0 = x$, Dynkin's formula says:
$$ \mathbb{E}[f(X_{\tau}) | X_0 = x] = f(x) + \mathbb{E}[\int_0^{\tau} A \bullet f(X_s) \cdot ds | X_0 = x]$$
\item Stochastic generalization of 2nd fundamental theorem of calculus
\end{itemize}
\end{frame}

\end{document}