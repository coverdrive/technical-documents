\documentclass[12pt]{exam}
\usepackage[utf8]{inputenc}
\usepackage{graphicx} % Allows including images
\usepackage{cool}
\usepackage{tikz}
\usepackage{amsmath}
\usepackage{listings}
\usepackage{pseudocode}
\usepackage[colorlinks = true,
            linkcolor = blue,
            urlcolor  = blue,
            citecolor = blue,
            anchorcolor = blue]{hyperref}
\usepackage{MnSymbol,wasysym}
\usepackage{geometry} % see geometry.pdf on how to lay out the page. There's lots.
\geometry{a4paper} 
\newgeometry{vmargin={20mm}, hmargin={14mm,18mm}}
 
\begin{document}
\begin{center}
{\large {\bf Stanford CME 241 (Winter 2022) - Assignment 9}}
\end{center}
 
{\large{\bf Assignments:}}
\begin{questions}
\question We'd like to build a simple simulator of Order Book Dynamics as a \lstinline{MarkovProcess} using the code in \href{https://github.com/TikhonJelvis/RL-book/blob/master/rl/chapter9/order_book.py}{rl\//chapter9\//order\_book.py}. An object of type \lstinline{OrderBook} constitutes the {\em State}. Your task is to come up with a simple model for random arrivals of Market Orders and Limit Orders based on the current contents of the \lstinline{OrderBook}. This model of random arrivals of Marker Orders and Limit Orders defines the probabilistic transitions from the current state (\lstinline{OrderBook} object) to the next state (\lstinline{OrderBook} object). Implement the probabilistic transitions as a \lstinline{MarkovProcess} and use it's \lstinline{simulate} method to complete your implementation of a simple simulator of Order Book Dynamics.

Experiment with different models for random arrivals of Market Orders and Limit Orders.

\question Derive the expressions for the Optimal Value Function and Optimal Policy for the {\em Linear-Percentage Temporary} (LPT) Price Impact Model formulated by Bertsimas and Lo. The LPT model is described below for all $t = 0, 1, \ldots T-1$:

$$P_{t+1} = P_t \cdot e^{Z_t}$$
$$X_{t+1} = \rho \cdot X_t + \eta_t$$
$$Q_t = P_t \cdot (1 - \beta \cdot N_t - \theta \cdot X_t)$$
where $Z_t$ are independent and identically distributed random variables with mean $\mu_Z$ and variance $\sigma^2_Z$ for all $t = 0, 1, \ldots, T-1$, $\eta_t$ are independent and identically distributed random variables with mean 0 for all $t = 0, 1, \ldots, T-1$, $Z_t$ and $\eta_t$ are independent of each other for all $t = 0, 1, \ldots, T-1$, and $\rho, \beta, \theta$ are given constants. The model assumes no risk-aversion (Utility function is the identity function) and so, the objective is to maximize the Expected Total Sales Proceeds over the finite-horizon up to time $T$ (discount factor is 1). In your derivation, use the same methodology as we followed for the {\em Simple Linear Price Impact Model with no Risk-Aversion}.

Implement this LPT model by customizing the class \lstinline{OptimalOrderExecution} in 

\href{https://github.com/TikhonJelvis/RL-book/blob/master/rl/chapter9/optimal_order_execution.py}{rl\//chapter9\//optimal\_order\_execution.py}.  

Compare the obtained Optimal Value Function and Optimal Policy against the closed-form solution you derived above.

\end{questions}

\end{document}