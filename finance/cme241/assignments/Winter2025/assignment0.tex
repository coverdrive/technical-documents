\documentclass[12pt]{exam}
\usepackage[utf8]{inputenc}
\usepackage{graphicx} % Allows including images
\usepackage{cool}
\usepackage{tikz}
\usepackage{amsmath}
\usepackage{listings}
\usepackage{pseudocode}
\usepackage[colorlinks = true,
            linkcolor = blue,
            urlcolor  = blue,
            citecolor = blue,
            anchorcolor = blue]{hyperref}
\usepackage{MnSymbol,wasysym}
\usepackage{geometry} % see geometry.pdf on how to lay out the page. There's lots.
\geometry{a4paper} 
\newgeometry{vmargin={20mm}, hmargin={14mm,18mm}}
 
\begin{document}
\begin{center}
{\large {\bf Stanford CME 241 Winter 2025 - Assignment 0 \\ due Sunday, January 12th @ 11:59 PM}}
\end{center}

\begin{questions}
\question Make sure you have access to the course Canvas and Ed Discussion pages \\ (email \href{mailto:neelsn@stanford.edu}{\underline{\textcolor{blue}{Neel Narayan}}} if you do not)
\question Install/Setup LaTeX/Markdown for technical writing and Python 3 for coding (optionally Jupyter, if you prefer working with .ipynb instead of plain .py).
\question {\em Fork} the \href{https://github.com/TikhonJelvis/RL-book}{Code Repo associated with the RLForFinanceBook} and get set up to write code (for future assignments) that uses classes/functions from this code repo, see below. Choose your favorite IDE or text editor.
\question Create clearly-named directories for assignments and the course project for the Course Assistant (\href{mailto:neelsn@stanford.edu}{\underline{\textcolor{blue}{Neel Narayan}}}) to review and grade - each team of up to 3 (who submits together) should send Neel their forked repo URL and {\em git push} their work by the due dates (to their forked repo).\footnote{We expect teams formed for assignments to remain consistent throughout the quarter. If issues arise, let us know.}
\question {\bf Optionally}, you can create the same virtual environment I use and replicate my dependencies with the following instructions:
\begin{itemize}
\item After forking the repo on your laptop, create a virtual environment with the following shell command (from the RL-book directory):
\begin{lstlisting}[language=bash]
$ python3 -m venv .venv
\end{lstlisting}
\item Then, each time you're working on this project, make sure to activate the venv with the following shell command (again, from the RL-book directory):
\begin{lstlisting}[language=bash]
$ source .venv/bin/activate
\end{lstlisting}
\item Once the venv is activated, you should see a (.venv) in your shell prompt
\item Now you can use pip to install dependencies inside the venv, for example:
\begin{lstlisting}[language=bash]
(.venv) $ pip install matplotlib
\end{lstlisting}
\item Initially, you can install every Python package you need to work this git repo with the following shell command (again, from the RL-book directory):
\begin{lstlisting}[language=bash]
(.venv) $ pip install -r requirements.txt
\end{lstlisting}
\item To work with the appropriate file paths of the Python files in this repo from the RL-book directory, execute the following command from the RL-book directory (this creates a package):
\begin{lstlisting}[language=bash]
(.venv) $ pip install -e .
\end{lstlisting}
\item To make sure you are all good, verify with the following command from the RL-book directory:
\begin{lstlisting}[language=bash]
(.venv) $ python -m unittest discover
\end{lstlisting}
If all is good, you should see an "OK" on the last line of the output upon running this command.
\item Some installations (e.g. Python 3.10) may run into build errors. Please check Ed for updated requirements.txt files or contact the Course Assistant if failures continue.
\end{itemize}
\end{questions}

\end{document}